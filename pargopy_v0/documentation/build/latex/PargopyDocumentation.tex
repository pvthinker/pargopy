%% Generated by Sphinx.
\def\sphinxdocclass{report}
\documentclass[letterpaper,10pt,english]{sphinxmanual}
\ifdefined\pdfpxdimen
   \let\sphinxpxdimen\pdfpxdimen\else\newdimen\sphinxpxdimen
\fi \sphinxpxdimen=.75bp\relax

\PassOptionsToPackage{warn}{textcomp}
\usepackage[utf8]{inputenc}
\ifdefined\DeclareUnicodeCharacter
 \ifdefined\DeclareUnicodeCharacterAsOptional
  \DeclareUnicodeCharacter{"00A0}{\nobreakspace}
  \DeclareUnicodeCharacter{"2500}{\sphinxunichar{2500}}
  \DeclareUnicodeCharacter{"2502}{\sphinxunichar{2502}}
  \DeclareUnicodeCharacter{"2514}{\sphinxunichar{2514}}
  \DeclareUnicodeCharacter{"251C}{\sphinxunichar{251C}}
  \DeclareUnicodeCharacter{"2572}{\textbackslash}
 \else
  \DeclareUnicodeCharacter{00A0}{\nobreakspace}
  \DeclareUnicodeCharacter{2500}{\sphinxunichar{2500}}
  \DeclareUnicodeCharacter{2502}{\sphinxunichar{2502}}
  \DeclareUnicodeCharacter{2514}{\sphinxunichar{2514}}
  \DeclareUnicodeCharacter{251C}{\sphinxunichar{251C}}
  \DeclareUnicodeCharacter{2572}{\textbackslash}
 \fi
\fi
\usepackage{cmap}
\usepackage[T1]{fontenc}
\usepackage{amsmath,amssymb,amstext}
\usepackage{babel}
\usepackage{times}
\usepackage[Bjarne]{fncychap}
\usepackage{sphinx}

\usepackage{geometry}

% Include hyperref last.
\usepackage{hyperref}
% Fix anchor placement for figures with captions.
\usepackage{hypcap}% it must be loaded after hyperref.
% Set up styles of URL: it should be placed after hyperref.
\urlstyle{same}

\addto\captionsenglish{\renewcommand{\figurename}{Fig.}}
\addto\captionsenglish{\renewcommand{\tablename}{Table}}
\addto\captionsenglish{\renewcommand{\literalblockname}{Listing}}

\addto\captionsenglish{\renewcommand{\literalblockcontinuedname}{continued from previous page}}
\addto\captionsenglish{\renewcommand{\literalblockcontinuesname}{continues on next page}}

\addto\extrasenglish{\def\pageautorefname{page}}

\setcounter{tocdepth}{1}



\title{Pargopy Documentation Documentation}
\date{May 22, 2018}
\release{}
\author{Thomas HERRY \& Guillaume ROULLET}
\newcommand{\sphinxlogo}{\vbox{}}
\renewcommand{\releasename}{}
\makeindex

\begin{document}

\maketitle
\sphinxtableofcontents
\phantomsection\label{\detokenize{index::doc}}



\chapter{pargopy package}
\label{\detokenize{pargopy:welcome-to-pargopy-s-documentation}}\label{\detokenize{pargopy:pargopy-package}}\label{\detokenize{pargopy::doc}}

\section{Submodules}
\label{\detokenize{pargopy:submodules}}

\section{argodb.py}
\label{\detokenize{pargopy:module-pargopy.argodb}}\label{\detokenize{pargopy:argodb-py}}\index{pargopy.argodb (module)}
Created on Mon Mar 12 13:10:24 2018

File creating the summary of ARGO used to generate the atlas
\index{get\_all\_wmos() (in module pargopy.argodb)}

\begin{fulllineitems}
\phantomsection\label{\detokenize{pargopy:pargopy.argodb.get_all_wmos}}\pysiglinewithargsret{\sphinxcode{\sphinxupquote{pargopy.argodb.}}\sphinxbfcode{\sphinxupquote{get\_all\_wmos}}}{}{}
Return a dictionnary of all wmo (list of int) with dac (string) as
keys HAVING a *\_prof.nc file

A few wmo have no *\_prof.nc (352 exactly) because … they have
actually no profile reported
\begin{quote}\begin{description}
\item[{Return type}] \leavevmode
dic

\end{description}\end{quote}

\end{fulllineitems}

\index{get\_header\_of\_all\_profiles() (in module pargopy.argodb)}

\begin{fulllineitems}
\phantomsection\label{\detokenize{pargopy:pargopy.argodb.get_header_of_all_profiles}}\pysiglinewithargsret{\sphinxcode{\sphinxupquote{pargopy.argodb.}}\sphinxbfcode{\sphinxupquote{get\_header\_of\_all\_profiles}}}{\emph{wmostats}}{}
Build argodb from the infos in wmostats

Once it is created it is more efficient to read it from the disk
using ‘read\_argodb()’
\begin{quote}\begin{description}
\item[{Return type}] \leavevmode
dic

\end{description}\end{quote}

\end{fulllineitems}

\index{get\_header\_of\_all\_wmos() (in module pargopy.argodb)}

\begin{fulllineitems}
\phantomsection\label{\detokenize{pargopy:pargopy.argodb.get_header_of_all_wmos}}\pysiglinewithargsret{\sphinxcode{\sphinxupquote{pargopy.argodb.}}\sphinxbfcode{\sphinxupquote{get\_header\_of\_all\_wmos}}}{\emph{wmodic}}{}
Get the header of all wmo
\begin{quote}\begin{description}
\item[{Return type}] \leavevmode
dic

\end{description}\end{quote}

\end{fulllineitems}

\index{main() (in module pargopy.argodb)}

\begin{fulllineitems}
\phantomsection\label{\detokenize{pargopy:pargopy.argodb.main}}\pysiglinewithargsret{\sphinxcode{\sphinxupquote{pargopy.argodb.}}\sphinxbfcode{\sphinxupquote{main}}}{}{}
Main function of argodb.py

\end{fulllineitems}

\index{propagate\_flag\_backward() (in module pargopy.argodb)}

\begin{fulllineitems}
\phantomsection\label{\detokenize{pargopy:pargopy.argodb.propagate_flag_backward}}\pysiglinewithargsret{\sphinxcode{\sphinxupquote{pargopy.argodb.}}\sphinxbfcode{\sphinxupquote{propagate\_flag\_backward}}}{\emph{argodb}, \emph{subargodb}, \emph{verbose=True}}{}
Update argodb FLAG using subargodb
:rtype: None

\end{fulllineitems}

\index{update\_wmodic() (in module pargopy.argodb)}

\begin{fulllineitems}
\phantomsection\label{\detokenize{pargopy:pargopy.argodb.update_wmodic}}\pysiglinewithargsret{\sphinxcode{\sphinxupquote{pargopy.argodb.}}\sphinxbfcode{\sphinxupquote{update\_wmodic}}}{}{}
Read the full argodb database and update argodb.pkl
\begin{quote}\begin{description}
\item[{Return type}] \leavevmode
None

\end{description}\end{quote}

\end{fulllineitems}



\section{argotools.py}
\label{\detokenize{pargopy:argotools-py}}\label{\detokenize{pargopy:module-pargopy.argotools}}\index{pargopy.argotools (module)}
Created on Mon Mar 12 13:10:24 2018

@author: herry

Tools used by all the python files to generate the atlas
\index{conversion\_gregd\_juld() (in module pargopy.argotools)}

\begin{fulllineitems}
\phantomsection\label{\detokenize{pargopy:pargopy.argotools.conversion_gregd_juld}}\pysiglinewithargsret{\sphinxcode{\sphinxupquote{pargopy.argotools.}}\sphinxbfcode{\sphinxupquote{conversion\_gregd\_juld}}}{\emph{year}, \emph{month}, \emph{day}}{}
Method converting gregorian day into julian day
\begin{quote}\begin{description}
\item[{Return type}] \leavevmode
float

\end{description}\end{quote}

\end{fulllineitems}

\index{conversion\_juld\_gregd() (in module pargopy.argotools)}

\begin{fulllineitems}
\phantomsection\label{\detokenize{pargopy:pargopy.argotools.conversion_juld_gregd}}\pysiglinewithargsret{\sphinxcode{\sphinxupquote{pargopy.argotools.}}\sphinxbfcode{\sphinxupquote{conversion\_juld\_gregd}}}{\emph{juld}}{}
Method converting julian day into gregorian day
\begin{quote}\begin{description}
\item[{Return type}] \leavevmode
list of int

\end{description}\end{quote}

\end{fulllineitems}

\index{count\_profiles\_in\_database() (in module pargopy.argotools)}

\begin{fulllineitems}
\phantomsection\label{\detokenize{pargopy:pargopy.argotools.count_profiles_in_database}}\pysiglinewithargsret{\sphinxcode{\sphinxupquote{pargopy.argotools.}}\sphinxbfcode{\sphinxupquote{count\_profiles\_in\_database}}}{\emph{wmostats}}{}
Count the total number of profiles in database
:rtype: int

\end{fulllineitems}

\index{count\_wmos() (in module pargopy.argotools)}

\begin{fulllineitems}
\phantomsection\label{\detokenize{pargopy:pargopy.argotools.count_wmos}}\pysiglinewithargsret{\sphinxcode{\sphinxupquote{pargopy.argotools.}}\sphinxbfcode{\sphinxupquote{count\_wmos}}}{\emph{wmodic}}{}
Count the total number of wmo in the Argo database base
:rtype: list of wmo

\end{fulllineitems}

\index{dac\_from\_wmo() (in module pargopy.argotools)}

\begin{fulllineitems}
\phantomsection\label{\detokenize{pargopy:pargopy.argotools.dac_from_wmo}}\pysiglinewithargsret{\sphinxcode{\sphinxupquote{pargopy.argotools.}}\sphinxbfcode{\sphinxupquote{dac\_from\_wmo}}}{\emph{wmodic}, \emph{wmo}}{}
Retrieve the dac of a wmo
:rtype: list of dac

\end{fulllineitems}

\index{extract\_idx\_from\_argodb() (in module pargopy.argotools)}

\begin{fulllineitems}
\phantomsection\label{\detokenize{pargopy:pargopy.argotools.extract_idx_from_argodb}}\pysiglinewithargsret{\sphinxcode{\sphinxupquote{pargopy.argotools.}}\sphinxbfcode{\sphinxupquote{extract\_idx\_from\_argodb}}}{\emph{argodb}, \emph{idx}}{}
Return a argodb type dictionnary that is a subset of argodb and
containing only entries given in idx (list)
\begin{quote}\begin{description}
\item[{Return type}] \leavevmode
dic

\end{description}\end{quote}

\end{fulllineitems}

\index{extract\_idx\_from\_wmostats() (in module pargopy.argotools)}

\begin{fulllineitems}
\phantomsection\label{\detokenize{pargopy:pargopy.argotools.extract_idx_from_wmostats}}\pysiglinewithargsret{\sphinxcode{\sphinxupquote{pargopy.argotools.}}\sphinxbfcode{\sphinxupquote{extract\_idx\_from\_wmostats}}}{\emph{wmostats}, \emph{idx}}{}
Return a wmostats type dictionnary that is a subset of wmostats and
containing only entries given in idx (list)
\begin{quote}\begin{description}
\item[{Return type}] \leavevmode
dic

\end{description}\end{quote}

\end{fulllineitems}

\index{extract\_idx\_inside\_tile() (in module pargopy.argotools)}

\begin{fulllineitems}
\phantomsection\label{\detokenize{pargopy:pargopy.argotools.extract_idx_inside_tile}}\pysiglinewithargsret{\sphinxcode{\sphinxupquote{pargopy.argotools.}}\sphinxbfcode{\sphinxupquote{extract\_idx\_inside\_tile}}}{\emph{res}, \emph{argodb}}{}
Extract from ‘argodb’ the list of profiles that are inside the tile
The tile limits are given by ‘res’
\begin{quote}
\begin{quote}\begin{description}
\item[{rtype}] \leavevmode
dic

\end{description}\end{quote}
\end{quote}

\end{fulllineitems}

\index{fix\_flag\_latlonf() (in module pargopy.argotools)}

\begin{fulllineitems}
\phantomsection\label{\detokenize{pargopy:pargopy.argotools.fix_flag_latlonf}}\pysiglinewithargsret{\sphinxcode{\sphinxupquote{pargopy.argotools.}}\sphinxbfcode{\sphinxupquote{fix\_flag\_latlonf}}}{\emph{argodb}}{}
Set a flag error for profiles having bad masked positions

Bad masked position yield value of 99999. This fix only concerns a
few profiles from dac=’jma’
\begin{quote}\begin{description}
\item[{Return type}] \leavevmode
None

\end{description}\end{quote}

\end{fulllineitems}

\index{flag\_argodb() (in module pargopy.argotools)}

\begin{fulllineitems}
\phantomsection\label{\detokenize{pargopy:pargopy.argotools.flag_argodb}}\pysiglinewithargsret{\sphinxcode{\sphinxupquote{pargopy.argotools.}}\sphinxbfcode{\sphinxupquote{flag\_argodb}}}{\emph{argodb}, \emph{wmodic}}{}
Add the flag to argodb
\begin{quote}\begin{description}
\item[{Return type}] \leavevmode
dic

\end{description}\end{quote}

\end{fulllineitems}

\index{get\_datamode() (in module pargopy.argotools)}

\begin{fulllineitems}
\phantomsection\label{\detokenize{pargopy:pargopy.argotools.get_datamode}}\pysiglinewithargsret{\sphinxcode{\sphinxupquote{pargopy.argotools.}}\sphinxbfcode{\sphinxupquote{get\_datamode}}}{\emph{data}}{}
Return the data mode of the profile
\begin{quote}\begin{description}
\item[{Return type}] \leavevmode
np.array

\end{description}\end{quote}

\end{fulllineitems}

\index{get\_idx\_from\_list\_wmo() (in module pargopy.argotools)}

\begin{fulllineitems}
\phantomsection\label{\detokenize{pargopy:pargopy.argotools.get_idx_from_list_wmo}}\pysiglinewithargsret{\sphinxcode{\sphinxupquote{pargopy.argotools.}}\sphinxbfcode{\sphinxupquote{get\_idx\_from\_list\_wmo}}}{\emph{argodb}, \emph{wmos}}{}
Get the list of profile indices present in argodb that correspond
to the list of wmos
\begin{quote}\begin{description}
\item[{Return type}] \leavevmode
list of int

\end{description}\end{quote}

\end{fulllineitems}

\index{get\_profile\_file\_path() (in module pargopy.argotools)}

\begin{fulllineitems}
\phantomsection\label{\detokenize{pargopy:pargopy.argotools.get_profile_file_path}}\pysiglinewithargsret{\sphinxcode{\sphinxupquote{pargopy.argotools.}}\sphinxbfcode{\sphinxupquote{get\_profile\_file\_path}}}{\emph{dac}, \emph{wmo}}{}
Return the file path to the *\_prof.nc data file
\begin{quote}\begin{description}
\item[{Return type}] \leavevmode
string

\end{description}\end{quote}

\end{fulllineitems}

\index{get\_tag() (in module pargopy.argotools)}

\begin{fulllineitems}
\phantomsection\label{\detokenize{pargopy:pargopy.argotools.get_tag}}\pysiglinewithargsret{\sphinxcode{\sphinxupquote{pargopy.argotools.}}\sphinxbfcode{\sphinxupquote{get\_tag}}}{\emph{kdac}, \emph{wmo}, \emph{kprof}}{}
Compute the tag number of a profile

The inverse of get\_tag() is retrieve\_infos\_from\_tag()
\begin{quote}\begin{description}
\item[{Return type}] \leavevmode
int

\end{description}\end{quote}

\end{fulllineitems}

\index{plot\_location\_profiles() (in module pargopy.argotools)}

\begin{fulllineitems}
\phantomsection\label{\detokenize{pargopy:pargopy.argotools.plot_location_profiles}}\pysiglinewithargsret{\sphinxcode{\sphinxupquote{pargopy.argotools.}}\sphinxbfcode{\sphinxupquote{plot\_location\_profiles}}}{\emph{argodb}}{}
Plot a scatter plot of profiles in argodb

argodb can be the full database or any subset
\begin{quote}\begin{description}
\item[{Return type}] \leavevmode
None

\end{description}\end{quote}

\end{fulllineitems}

\index{plot\_wmo\_data() (in module pargopy.argotools)}

\begin{fulllineitems}
\phantomsection\label{\detokenize{pargopy:pargopy.argotools.plot_wmo_data}}\pysiglinewithargsret{\sphinxcode{\sphinxupquote{pargopy.argotools.}}\sphinxbfcode{\sphinxupquote{plot\_wmo\_data}}}{\emph{dac}, \emph{wmo}}{}
Plot raw ‘TEMP’ data for dac, wmo
\begin{quote}\begin{description}
\item[{Return type}] \leavevmode
None

\end{description}\end{quote}

\end{fulllineitems}

\index{plot\_wmos\_stats() (in module pargopy.argotools)}

\begin{fulllineitems}
\phantomsection\label{\detokenize{pargopy:pargopy.argotools.plot_wmos_stats}}\pysiglinewithargsret{\sphinxcode{\sphinxupquote{pargopy.argotools.}}\sphinxbfcode{\sphinxupquote{plot\_wmos\_stats}}}{\emph{wmostats}}{}
Plot the histogram of number of profiles per number of levels
\begin{quote}\begin{description}
\item[{Return type}] \leavevmode
None

\end{description}\end{quote}

\end{fulllineitems}

\index{read\_dic() (in module pargopy.argotools)}

\begin{fulllineitems}
\phantomsection\label{\detokenize{pargopy:pargopy.argotools.read_dic}}\pysiglinewithargsret{\sphinxcode{\sphinxupquote{pargopy.argotools.}}\sphinxbfcode{\sphinxupquote{read\_dic}}}{\emph{name}, \emph{path\_localdata}}{}
Function used to read each dic used to create .pkl files
Regroups :
- read\_wmodic, read\_wmstats, read\_argodb from argodb.py
- read\_argo\_filter from research\_tools.py
- read\_tile from tile.py
\begin{quote}\begin{description}
\item[{Return type}] \leavevmode
dict

\end{description}\end{quote}

\end{fulllineitems}

\index{read\_profile() (in module pargopy.argotools)}

\begin{fulllineitems}
\phantomsection\label{\detokenize{pargopy:pargopy.argotools.read_profile}}\pysiglinewithargsret{\sphinxcode{\sphinxupquote{pargopy.argotools.}}\sphinxbfcode{\sphinxupquote{read\_profile}}}{\emph{dac}, \emph{wmo}, \emph{iprof=None}, \emph{header=False}, \emph{data=False}, \emph{headerqc=False}, \emph{dataqc=False}, \emph{verbose=True}}{}
Basic driver to read the *\_prof.nc data file

The output is a dictionnary of vectors
- read one or all profiles read the header (lat, lon, juld) or not
- read the data or not always return IDAC, WMO, N\_PROF, N\_LEVELS
- and DATA\_UPDATE (all 5 are int)
\begin{quote}\begin{description}
\item[{Return type}] \leavevmode
dic

\end{description}\end{quote}

\end{fulllineitems}

\index{retrieve\_infos\_from\_tag() (in module pargopy.argotools)}

\begin{fulllineitems}
\phantomsection\label{\detokenize{pargopy:pargopy.argotools.retrieve_infos_from_tag}}\pysiglinewithargsret{\sphinxcode{\sphinxupquote{pargopy.argotools.}}\sphinxbfcode{\sphinxupquote{retrieve\_infos\_from\_tag}}}{\emph{argodb}, \emph{tag}}{}
Retrieve idac, wmo and iprof from tag (array of int)

It is the inverse of get\_tag()
\begin{quote}\begin{description}
\item[{Return type}] \leavevmode
dic

\end{description}\end{quote}

\end{fulllineitems}

\index{test\_tiles() (in module pargopy.argotools)}

\begin{fulllineitems}
\phantomsection\label{\detokenize{pargopy:pargopy.argotools.test_tiles}}\pysiglinewithargsret{\sphinxcode{\sphinxupquote{pargopy.argotools.}}\sphinxbfcode{\sphinxupquote{test\_tiles}}}{\emph{argo}, \emph{i}}{}
Test that tile ‘i’ is correctly defined with all profiles positions
within the tile limits, encompassing the margins
\begin{quote}\begin{description}
\item[{Return type}] \leavevmode
None

\end{description}\end{quote}

\end{fulllineitems}

\index{tile\_definition() (in module pargopy.argotools)}

\begin{fulllineitems}
\phantomsection\label{\detokenize{pargopy:pargopy.argotools.tile_definition}}\pysiglinewithargsret{\sphinxcode{\sphinxupquote{pargopy.argotools.}}\sphinxbfcode{\sphinxupquote{tile\_definition}}}{}{}
Define the tiles coordinates, in the form of a vector of lon and
lat + their margins

The tile indexing is

{\color{red}\bfseries{}\textbar{}-----+-----+-----+-----+-----\textbar{}}
\textbar{} 280 \textbar{} 281 \textbar{} 282 \textbar{} … \textbar{} 299 \textbar{}
{\color{red}\bfseries{}\textbar{}-----+-----+-----+-----+-----\textbar{}}
\textbar{} … \textbar{} … \textbar{} … \textbar{} … \textbar{} … \textbar{}
{\color{red}\bfseries{}\textbar{}-----+-----+-----+-----+-----\textbar{}}
\textbar{}  20 \textbar{}  21 \textbar{}  22 \textbar{} … \textbar{}  39 \textbar{}
{\color{red}\bfseries{}\textbar{}-----+-----+-----+-----+-----\textbar{}}
\textbar{}   0 \textbar{}   1 \textbar{}   2 \textbar{} … \textbar{}  19 \textbar{}
{\color{red}\bfseries{}\textbar{}-----+-----+-----+-----+-----\textbar{}}
\begin{quote}
\begin{quote}\begin{description}
\item[{rtype}] \leavevmode
float, float, int, int, float, float

\end{description}\end{quote}
\end{quote}

\end{fulllineitems}

\index{write\_dic() (in module pargopy.argotools)}

\begin{fulllineitems}
\phantomsection\label{\detokenize{pargopy:pargopy.argotools.write_dic}}\pysiglinewithargsret{\sphinxcode{\sphinxupquote{pargopy.argotools.}}\sphinxbfcode{\sphinxupquote{write\_dic}}}{\emph{name}, \emph{dic}, \emph{path\_localdata}}{}
Function used to write each dic used to create .pkl files
Regroups :
- write\_wmodic, write\_wmstats, write\_argodb from argodb.py
- write\_argo\_filter from research\_tools.py
- write\_tile from tile.py
\begin{quote}\begin{description}
\item[{Return type}] \leavevmode
None

\end{description}\end{quote}

\end{fulllineitems}



\section{atlas.py}
\label{\detokenize{pargopy:atlas-py}}\label{\detokenize{pargopy:module-pargopy.atlas}}\index{pargopy.atlas (module)}
Created on Mon Mar 12 13:10:24 2018
File creating the atlas of stats
from netCDF4  import Dataset
import numpy as np
import os
import argotools as argotools
import param as param
\index{atlas\_filename() (in module pargopy.atlas)}

\begin{fulllineitems}
\phantomsection\label{\detokenize{pargopy:pargopy.atlas.atlas_filename}}\pysiglinewithargsret{\sphinxcode{\sphinxupquote{pargopy.atlas.}}\sphinxbfcode{\sphinxupquote{atlas\_filename}}}{\emph{diratlas}, \emph{reso}, \emph{year}, \emph{mode}, \emph{typestat}}{}
\end{fulllineitems}

\index{get\_glo\_grid() (in module pargopy.atlas)}

\begin{fulllineitems}
\phantomsection\label{\detokenize{pargopy:pargopy.atlas.get_glo_grid}}\pysiglinewithargsret{\sphinxcode{\sphinxupquote{pargopy.atlas.}}\sphinxbfcode{\sphinxupquote{get\_glo\_grid}}}{\emph{reso}}{}
\end{fulllineitems}

\index{glue\_tiles() (in module pargopy.atlas)}

\begin{fulllineitems}
\phantomsection\label{\detokenize{pargopy:pargopy.atlas.glue_tiles}}\pysiglinewithargsret{\sphinxcode{\sphinxupquote{pargopy.atlas.}}\sphinxbfcode{\sphinxupquote{glue\_tiles}}}{\emph{reso}}{}
Glue the stats tiles together into a global 3D atlas

\end{fulllineitems}

\index{gridindex2lonlat() (in module pargopy.atlas)}

\begin{fulllineitems}
\phantomsection\label{\detokenize{pargopy:pargopy.atlas.gridindex2lonlat}}\pysiglinewithargsret{\sphinxcode{\sphinxupquote{pargopy.atlas.}}\sphinxbfcode{\sphinxupquote{gridindex2lonlat}}}{\emph{ix}, \emph{iy}}{}
\end{fulllineitems}

\index{ij2tile() (in module pargopy.atlas)}

\begin{fulllineitems}
\phantomsection\label{\detokenize{pargopy:pargopy.atlas.ij2tile}}\pysiglinewithargsret{\sphinxcode{\sphinxupquote{pargopy.atlas.}}\sphinxbfcode{\sphinxupquote{ij2tile}}}{\emph{i}, \emph{j}}{}
\end{fulllineitems}



\section{check\_decreasing\_pressure.py}
\label{\detokenize{pargopy:module-pargopy.check_decreasing_pressure}}\label{\detokenize{pargopy:check-decreasing-pressure-py}}\index{pargopy.check\_decreasing\_pressure (module)}
Created on Mon Mar 12 13:10:24 2018

File used to avoid values error for pressure, temperature and salinity
\index{check\_pressure() (in module pargopy.check\_decreasing\_pressure)}

\begin{fulllineitems}
\phantomsection\label{\detokenize{pargopy:pargopy.check_decreasing_pressure.check_pressure}}\pysiglinewithargsret{\sphinxcode{\sphinxupquote{pargopy.check\_decreasing\_pressure.}}\sphinxbfcode{\sphinxupquote{check\_pressure}}}{\emph{p}}{}
\end{fulllineitems}

\index{try\_to\_remove\_duplicate\_pressure() (in module pargopy.check\_decreasing\_pressure)}

\begin{fulllineitems}
\phantomsection\label{\detokenize{pargopy:pargopy.check_decreasing_pressure.try_to_remove_duplicate_pressure}}\pysiglinewithargsret{\sphinxcode{\sphinxupquote{pargopy.check\_decreasing\_pressure.}}\sphinxbfcode{\sphinxupquote{try\_to\_remove\_duplicate\_pressure}}}{\emph{p}}{}
\end{fulllineitems}



\section{decorator.py}
\label{\detokenize{pargopy:module-pargopy.decorator}}\label{\detokenize{pargopy:decorator-py}}\index{pargopy.decorator (module)}
Created on Thu Apr 26 07:16:12 2018

@author: therry
\index{call\_results() (in module pargopy.decorator)}

\begin{fulllineitems}
\phantomsection\label{\detokenize{pargopy:pargopy.decorator.call_results}}\pysiglinewithargsret{\sphinxcode{\sphinxupquote{pargopy.decorator.}}\sphinxbfcode{\sphinxupquote{call\_results}}}{\emph{filename}}{}
Function used to know the number of call and the mean time of execution
of each function from a chosen file
Do not forget to import the file you want to analyse 
before using this function !

\end{fulllineitems}

\index{exec\_time (class in pargopy.decorator)}

\begin{fulllineitems}
\phantomsection\label{\detokenize{pargopy:pargopy.decorator.exec_time}}\pysiglinewithargsret{\sphinxbfcode{\sphinxupquote{class }}\sphinxcode{\sphinxupquote{pargopy.decorator.}}\sphinxbfcode{\sphinxupquote{exec\_time}}}{\emph{fonc}}{}
Bases: \sphinxcode{\sphinxupquote{object}}

Decorator used to know the number of call for a function
and the Mean time for its execution
\index{results() (pargopy.decorator.exec\_time method)}

\begin{fulllineitems}
\phantomsection\label{\detokenize{pargopy:pargopy.decorator.exec_time.results}}\pysiglinewithargsret{\sphinxbfcode{\sphinxupquote{results}}}{}{}
Print the result :
- Number of call
- Mean time for execution

\end{fulllineitems}


\end{fulllineitems}



\section{general\_tools.py}
\label{\detokenize{pargopy:general-tools-py}}\label{\detokenize{pargopy:module-pargopy.general_tools}}\index{pargopy.general\_tools (module)}\index{compute\_weight() (in module pargopy.general\_tools)}

\begin{fulllineitems}
\phantomsection\label{\detokenize{pargopy:pargopy.general_tools.compute_weight}}\pysiglinewithargsret{\sphinxcode{\sphinxupquote{pargopy.general\_tools.}}\sphinxbfcode{\sphinxupquote{compute\_weight}}}{\emph{x}, \emph{y}, \emph{lon}, \emph{lat}, \emph{reso}}{}
Compute the weight between points (x, y) and point (lon, lat) with
a gaussian filter

\end{fulllineitems}

\index{cubiccoef() (in module pargopy.general\_tools)}

\begin{fulllineitems}
\phantomsection\label{\detokenize{pargopy:pargopy.general_tools.cubiccoef}}\pysiglinewithargsret{\sphinxcode{\sphinxupquote{pargopy.general\_tools.}}\sphinxbfcode{\sphinxupquote{cubiccoef}}}{\emph{z0}, \emph{zs}}{}
Weights for cubic interpolation at z0 given the four depths in zs

\end{fulllineitems}

\index{deg\_to\_rad() (in module pargopy.general\_tools)}

\begin{fulllineitems}
\phantomsection\label{\detokenize{pargopy:pargopy.general_tools.deg_to_rad}}\pysiglinewithargsret{\sphinxcode{\sphinxupquote{pargopy.general\_tools.}}\sphinxbfcode{\sphinxupquote{deg\_to\_rad}}}{\emph{angle}}{}
Degree to radians

\end{fulllineitems}

\index{dist\_sphe() (in module pargopy.general\_tools)}

\begin{fulllineitems}
\phantomsection\label{\detokenize{pargopy:pargopy.general_tools.dist_sphe}}\pysiglinewithargsret{\sphinxcode{\sphinxupquote{pargopy.general\_tools.}}\sphinxbfcode{\sphinxupquote{dist\_sphe}}}{\emph{x}, \emph{y}, \emph{lon}, \emph{lat}}{}
Compute the spherical arc between two points on the unit sphere

\end{fulllineitems}

\index{fixqcarray() (in module pargopy.general\_tools)}

\begin{fulllineitems}
\phantomsection\label{\detokenize{pargopy:pargopy.general_tools.fixqcarray}}\pysiglinewithargsret{\sphinxcode{\sphinxupquote{pargopy.general\_tools.}}\sphinxbfcode{\sphinxupquote{fixqcarray}}}{\emph{qc}}{}
Transform a Argo 2D array of qc flags (string) into a int array

\end{fulllineitems}

\index{insitu\_to\_absolute() (in module pargopy.general\_tools)}

\begin{fulllineitems}
\phantomsection\label{\detokenize{pargopy:pargopy.general_tools.insitu_to_absolute}}\pysiglinewithargsret{\sphinxcode{\sphinxupquote{pargopy.general\_tools.}}\sphinxbfcode{\sphinxupquote{insitu\_to\_absolute}}}{\emph{Tis}, \emph{SP}, \emph{p}, \emph{lon}, \emph{lat}, \emph{zref}}{}
Transform in situ variables to TEOS10 variables

\end{fulllineitems}

\index{interp\_at\_zref() (in module pargopy.general\_tools)}

\begin{fulllineitems}
\phantomsection\label{\detokenize{pargopy:pargopy.general_tools.interp_at_zref}}\pysiglinewithargsret{\sphinxcode{\sphinxupquote{pargopy.general\_tools.}}\sphinxbfcode{\sphinxupquote{interp\_at\_zref}}}{\emph{CT}, \emph{SA}, \emph{z}, \emph{zref}}{}
Interpolate CT and SA from their native depths z to zref

\end{fulllineitems}

\index{lincoef() (in module pargopy.general\_tools)}

\begin{fulllineitems}
\phantomsection\label{\detokenize{pargopy:pargopy.general_tools.lincoef}}\pysiglinewithargsret{\sphinxcode{\sphinxupquote{pargopy.general\_tools.}}\sphinxbfcode{\sphinxupquote{lincoef}}}{\emph{z0}, \emph{zs}}{}
Weights for linear interpolation at z0 given the two depths in zs

\end{fulllineitems}

\index{npa2ma() (in module pargopy.general\_tools)}

\begin{fulllineitems}
\phantomsection\label{\detokenize{pargopy:pargopy.general_tools.npa2ma}}\pysiglinewithargsret{\sphinxcode{\sphinxupquote{pargopy.general\_tools.}}\sphinxbfcode{\sphinxupquote{npa2ma}}}{\emph{x}}{}
Convert a numpy array into a masked array
set mask=True on NaN

\end{fulllineitems}

\index{raw\_to\_interpolate() (in module pargopy.general\_tools)}

\begin{fulllineitems}
\phantomsection\label{\detokenize{pargopy:pargopy.general_tools.raw_to_interpolate}}\pysiglinewithargsret{\sphinxcode{\sphinxupquote{pargopy.general\_tools.}}\sphinxbfcode{\sphinxupquote{raw\_to\_interpolate}}}{\emph{temp}, \emph{sal}, \emph{pres}, \emph{temp\_qc}, \emph{sal\_qc}, \emph{pres\_qc}, \emph{lon}, \emph{lat}, \emph{zref}}{}
Interpolate in situ data on zref depths
ierr = 0: no pb
ierr \textgreater{} 0: pb

\end{fulllineitems}

\index{remove\_bad\_qc() (in module pargopy.general\_tools)}

\begin{fulllineitems}
\phantomsection\label{\detokenize{pargopy:pargopy.general_tools.remove_bad_qc}}\pysiglinewithargsret{\sphinxcode{\sphinxupquote{pargopy.general\_tools.}}\sphinxbfcode{\sphinxupquote{remove\_bad\_qc}}}{\emph{temp}, \emph{sal}, \emph{pres}, \emph{temp\_qc}, \emph{sal\_qc}, \emph{pres\_qc}}{}
Return the index list of data for which the three qc’s are 1
and the error flag ierr
ierr = 0 : no pb
ierr = 1 : too few data in the profile

\end{fulllineitems}

\index{select\_depth() (in module pargopy.general\_tools)}

\begin{fulllineitems}
\phantomsection\label{\detokenize{pargopy:pargopy.general_tools.select_depth}}\pysiglinewithargsret{\sphinxcode{\sphinxupquote{pargopy.general\_tools.}}\sphinxbfcode{\sphinxupquote{select\_depth}}}{\emph{zref}, \emph{z}}{}
Return the number of data points we have between successive zref.
This is used to decide which interpolation is the best: none,
linear or cubic. For endpoints (zref=0) and bottom(zref=2000) use
linear extrapolation

\end{fulllineitems}



\section{interpolation\_tools.py}
\label{\detokenize{pargopy:interpolation-tools-py}}\label{\detokenize{pargopy:module-pargopy.interpolation_tools}}\index{pargopy.interpolation\_tools (module)}
Created on Wed Mar 14 14:37:02 2018

@author: herry

Tools used for the interpolation of the values from ARGO
\index{insitu\_to\_absolute() (in module pargopy.interpolation\_tools)}

\begin{fulllineitems}
\phantomsection\label{\detokenize{pargopy:pargopy.interpolation_tools.insitu_to_absolute}}\pysiglinewithargsret{\sphinxcode{\sphinxupquote{pargopy.interpolation\_tools.}}\sphinxbfcode{\sphinxupquote{insitu\_to\_absolute}}}{\emph{Tis}, \emph{SP}, \emph{p}, \emph{lon}, \emph{lat}, \emph{zref}}{}
Transform in situ variables to TEOS10 variables
\begin{quote}\begin{description}
\item[{Return type}] \leavevmode
float, float, float

\end{description}\end{quote}

\end{fulllineitems}

\index{interp\_at\_zref() (in module pargopy.interpolation\_tools)}

\begin{fulllineitems}
\phantomsection\label{\detokenize{pargopy:pargopy.interpolation_tools.interp_at_zref}}\pysiglinewithargsret{\sphinxcode{\sphinxupquote{pargopy.interpolation\_tools.}}\sphinxbfcode{\sphinxupquote{interp\_at\_zref}}}{\emph{CT}, \emph{SA}, \emph{z}, \emph{zref}}{}
Interpolate CT, SA, dCT/dz and dSA/dz from their native depths z to
zref

Method: we use piecewise Lagrange polynomial interpolation

For each zref{[}k{]}, we select a list of z{[}j{]} that are close to
zref{[}k{]}, imposing to have z{[}j{]} that are above and below zref{[}k{]}
(except near the boundaries)

If only two z{[}j{]} are found then the result is a linear interpolation

If n z{[}j{]} are found then the result is a n-th order interpolation.

For interior points we may go up to 6-th order

For the surface level (zref==0), we do extrapolation

For the bottom level (zref=2000), we do either extrapolation or
interpolation if data deeper than 2000 are available.
\begin{quote}\begin{description}
\item[{Return type}] \leavevmode
float, float, float, float

\end{description}\end{quote}

\end{fulllineitems}

\index{interpolate\_profiles() (in module pargopy.interpolation\_tools)}

\begin{fulllineitems}
\phantomsection\label{\detokenize{pargopy:pargopy.interpolation_tools.interpolate_profiles}}\pysiglinewithargsret{\sphinxcode{\sphinxupquote{pargopy.interpolation\_tools.}}\sphinxbfcode{\sphinxupquote{interpolate\_profiles}}}{\emph{subargodb}, \emph{wmodic}}{}
Interpolate the profiles in subargodb
\begin{quote}\begin{description}
\item[{Return type}] \leavevmode
dic

\end{description}\end{quote}

\end{fulllineitems}

\index{lagrangepoly() (in module pargopy.interpolation\_tools)}

\begin{fulllineitems}
\phantomsection\label{\detokenize{pargopy:pargopy.interpolation_tools.lagrangepoly}}\pysiglinewithargsret{\sphinxcode{\sphinxupquote{pargopy.interpolation\_tools.}}\sphinxbfcode{\sphinxupquote{lagrangepoly}}}{\emph{x0}, \emph{xi}}{}
Weights for polynomial interpolation at x0 given a list of xi
return both the weights for function (cs) and its first derivative
(ds)

Example:
lagrangepoly(0.25, {[}0, 1{]})
\textgreater{}\textgreater{}\textgreater{} {[}0.75, 0.25,{]}, {[}1, -1{]}
\begin{quote}\begin{description}
\item[{Return type}] \leavevmode
float, float

\end{description}\end{quote}

\end{fulllineitems}

\index{raw\_to\_interpolate() (in module pargopy.interpolation\_tools)}

\begin{fulllineitems}
\phantomsection\label{\detokenize{pargopy:pargopy.interpolation_tools.raw_to_interpolate}}\pysiglinewithargsret{\sphinxcode{\sphinxupquote{pargopy.interpolation\_tools.}}\sphinxbfcode{\sphinxupquote{raw\_to\_interpolate}}}{\emph{temp}, \emph{sal}, \emph{pres}, \emph{temp\_qc}, \emph{sal\_qc}, \emph{pres\_qc}, \emph{lon}, \emph{lat}, \emph{zref}}{}
Interpolate in situ data on zref depths

ierr = 0: no pb

ierr \textgreater{} 0: pb
\begin{quote}\begin{description}
\item[{Return type}] \leavevmode
float, float, float, float, float, float, float, int

\end{description}\end{quote}

\end{fulllineitems}

\index{remove\_bad\_qc() (in module pargopy.interpolation\_tools)}

\begin{fulllineitems}
\phantomsection\label{\detokenize{pargopy:pargopy.interpolation_tools.remove_bad_qc}}\pysiglinewithargsret{\sphinxcode{\sphinxupquote{pargopy.interpolation\_tools.}}\sphinxbfcode{\sphinxupquote{remove\_bad\_qc}}}{\emph{temp}, \emph{sal}, \emph{pres}, \emph{temp\_qc}, \emph{sal\_qc}, \emph{pres\_qc}}{}
Return the index list of data for which the three qc’s are 1
and the error flag ierr

ierr = 0 : no pb

ierr = 1 : too few data in the profile
\begin{quote}\begin{description}
\item[{Return type}] \leavevmode
list, int

\end{description}\end{quote}

\end{fulllineitems}

\index{select\_depth() (in module pargopy.interpolation\_tools)}

\begin{fulllineitems}
\phantomsection\label{\detokenize{pargopy:pargopy.interpolation_tools.select_depth}}\pysiglinewithargsret{\sphinxcode{\sphinxupquote{pargopy.interpolation\_tools.}}\sphinxbfcode{\sphinxupquote{select\_depth}}}{\emph{zref}, \emph{z}}{}
Return the number of data points we have between successive zref.

for each intervale k, we select the z\_j such that

zref{[}k{]} \textless{}= z\_j \textless{} zref{[}k+1{]}, for k=0 .. nref-2

zref{[}nref-1{]} \textless{}= z\_j \textless{} zextra, for k=nref-1

and return

nbperintervale{[}k{]} = number of z\_j

kperint{[}k{]} = list of j’s

with zextra = 2*zref{[}-1{]} - zref{[}-2{]}
\begin{quote}\begin{description}
\item[{Return type}] \leavevmode
int, list

\end{description}\end{quote}

\end{fulllineitems}



\section{param.py}
\label{\detokenize{pargopy:module-pargopy.param}}\label{\detokenize{pargopy:param-py}}\index{pargopy.param (module)}
Created on Wed Apr 11 11:01:25 2018

@author: herry

File defining the pathes to use this program


\section{research\_tools.py}
\label{\detokenize{pargopy:module-pargopy.research_tools}}\label{\detokenize{pargopy:research-tools-py}}\index{pargopy.research\_tools (module)}
Created on Tue Mar 20 15:24:20 2018

@author: herry

Tools used to generate the filters that are used after to generate the tiles
\index{creating\_tiles() (in module pargopy.research\_tools)}

\begin{fulllineitems}
\phantomsection\label{\detokenize{pargopy:pargopy.research_tools.creating_tiles}}\pysiglinewithargsret{\sphinxcode{\sphinxupquote{pargopy.research\_tools.}}\sphinxbfcode{\sphinxupquote{creating\_tiles}}}{}{}
Giving values to the variables
\begin{quote}\begin{description}
\item[{Return type}] \leavevmode
None

\end{description}\end{quote}

\end{fulllineitems}

\index{extract\_idx\_from\_argodb() (in module pargopy.research\_tools)}

\begin{fulllineitems}
\phantomsection\label{\detokenize{pargopy:pargopy.research_tools.extract_idx_from_argodb}}\pysiglinewithargsret{\sphinxcode{\sphinxupquote{pargopy.research\_tools.}}\sphinxbfcode{\sphinxupquote{extract\_idx\_from\_argodb}}}{\emph{argodb}, \emph{idx}}{}
Return a argodb type dictionnary that is a subset of argodb and
containing only entries given in idx (list)
\begin{quote}\begin{description}
\item[{Return type}] \leavevmode
dic

\end{description}\end{quote}

\end{fulllineitems}

\index{extract\_idx\_from\_wmostats() (in module pargopy.research\_tools)}

\begin{fulllineitems}
\phantomsection\label{\detokenize{pargopy:pargopy.research_tools.extract_idx_from_wmostats}}\pysiglinewithargsret{\sphinxcode{\sphinxupquote{pargopy.research\_tools.}}\sphinxbfcode{\sphinxupquote{extract\_idx\_from\_wmostats}}}{\emph{wmostats}, \emph{idx}}{}
Return a wmostats type dictionnary that is a subset of wmostats and
containing only entries given in idx (list)
\begin{quote}\begin{description}
\item[{Return type}] \leavevmode
dic

\end{description}\end{quote}

\end{fulllineitems}

\index{get\_idx\_from\_list\_wmo() (in module pargopy.research\_tools)}

\begin{fulllineitems}
\phantomsection\label{\detokenize{pargopy:pargopy.research_tools.get_idx_from_list_wmo}}\pysiglinewithargsret{\sphinxcode{\sphinxupquote{pargopy.research\_tools.}}\sphinxbfcode{\sphinxupquote{get\_idx\_from\_list\_wmo}}}{\emph{argodb}, \emph{wmos}}{}
Get the list of profile indices present in argodb that correspond
to the list of wmos
\begin{quote}\begin{description}
\item[{Return type}] \leavevmode
list of int

\end{description}\end{quote}

\end{fulllineitems}

\index{get\_idx\_from\_tiles\_lim() (in module pargopy.research\_tools)}

\begin{fulllineitems}
\phantomsection\label{\detokenize{pargopy:pargopy.research_tools.get_idx_from_tiles_lim}}\pysiglinewithargsret{\sphinxcode{\sphinxupquote{pargopy.research\_tools.}}\sphinxbfcode{\sphinxupquote{get\_idx\_from\_tiles\_lim}}}{\emph{res}, \emph{argodb}}{}
Get the list of profile indices present in argodb that correspond
to the list of wmos
\begin{quote}\begin{description}
\item[{Return type}] \leavevmode
dic

\end{description}\end{quote}

\end{fulllineitems}

\index{main() (in module pargopy.research\_tools)}

\begin{fulllineitems}
\phantomsection\label{\detokenize{pargopy:pargopy.research_tools.main}}\pysiglinewithargsret{\sphinxcode{\sphinxupquote{pargopy.research\_tools.}}\sphinxbfcode{\sphinxupquote{main}}}{}{}
Main function of stats.py

\end{fulllineitems}

\index{mbox() (in module pargopy.research\_tools)}

\begin{fulllineitems}
\phantomsection\label{\detokenize{pargopy:pargopy.research_tools.mbox}}\pysiglinewithargsret{\sphinxcode{\sphinxupquote{pargopy.research\_tools.}}\sphinxbfcode{\sphinxupquote{mbox}}}{\emph{x1}, \emph{x2}, \emph{y1}, \emph{y2}, \emph{dlat}, \emph{dlon}, \emph{col}, \emph{itile}, \emph{m}}{}
\end{fulllineitems}

\index{plot\_map() (in module pargopy.research\_tools)}

\begin{fulllineitems}
\phantomsection\label{\detokenize{pargopy:pargopy.research_tools.plot_map}}\pysiglinewithargsret{\sphinxcode{\sphinxupquote{pargopy.research\_tools.}}\sphinxbfcode{\sphinxupquote{plot\_map}}}{}{}
\end{fulllineitems}

\index{test\_tiles() (in module pargopy.research\_tools)}

\begin{fulllineitems}
\phantomsection\label{\detokenize{pargopy:pargopy.research_tools.test_tiles}}\pysiglinewithargsret{\sphinxcode{\sphinxupquote{pargopy.research\_tools.}}\sphinxbfcode{\sphinxupquote{test\_tiles}}}{\emph{argo\_extract}, \emph{i}}{}
Test to know if the tiles are correctly done with the lat and lon 
limits
\begin{quote}\begin{description}
\item[{Return type}] \leavevmode
None

\end{description}\end{quote}

\end{fulllineitems}



\section{stats.py}
\label{\detokenize{pargopy:module-pargopy.stats}}\label{\detokenize{pargopy:stats-py}}\index{pargopy.stats (module)}
Compute statistics on one tile
\index{compute\_mean\_at\_zref() (in module pargopy.stats)}

\begin{fulllineitems}
\phantomsection\label{\detokenize{pargopy:pargopy.stats.compute_mean_at_zref}}\pysiglinewithargsret{\sphinxcode{\sphinxupquote{pargopy.stats.}}\sphinxbfcode{\sphinxupquote{compute\_mean\_at\_zref}}}{\emph{itile}, \emph{reso\_deg}, \emph{mode}, \emph{date}}{}
Compute the mean at depths zref
\begin{quote}\begin{description}
\item[{Return type}] \leavevmode
dict

\end{description}\end{quote}

\end{fulllineitems}

\index{compute\_stats\_at\_zref() (in module pargopy.stats)}

\begin{fulllineitems}
\phantomsection\label{\detokenize{pargopy:pargopy.stats.compute_stats_at_zref}}\pysiglinewithargsret{\sphinxcode{\sphinxupquote{pargopy.stats.}}\sphinxbfcode{\sphinxupquote{compute\_stats\_at\_zref}}}{\emph{mode}, \emph{date}, \emph{grid\_lon}, \emph{grid\_lat}, \emph{reso\_deg}}{}
compute statistics on a small grid defined at grid\_lon x grid\_lat
the small grid should fit inside one tile

\end{fulllineitems}

\index{compute\_std\_at\_zref() (in module pargopy.stats)}

\begin{fulllineitems}
\phantomsection\label{\detokenize{pargopy:pargopy.stats.compute_std_at_zref}}\pysiglinewithargsret{\sphinxcode{\sphinxupquote{pargopy.stats.}}\sphinxbfcode{\sphinxupquote{compute\_std\_at\_zref}}}{\emph{itile}, \emph{reso\_deg}, \emph{timeflag}, \emph{mode}, \emph{date}, \emph{verbose=False}}{}
Compute the standard deviations at depths zref
\begin{quote}\begin{description}
\item[{Return type}] \leavevmode
dict

\end{description}\end{quote}

\end{fulllineitems}

\index{create\_stat\_file() (in module pargopy.stats)}

\begin{fulllineitems}
\phantomsection\label{\detokenize{pargopy:pargopy.stats.create_stat_file}}\pysiglinewithargsret{\sphinxcode{\sphinxupquote{pargopy.stats.}}\sphinxbfcode{\sphinxupquote{create\_stat\_file}}}{\emph{itile}, \emph{typestat}, \emph{reso}, \emph{timeflag}, \emph{date}, \emph{mode}}{}
Create statistics netcdf file
\begin{quote}\begin{description}
\item[{Return type}] \leavevmode
None

\end{description}\end{quote}

\end{fulllineitems}

\index{date\_mode\_filter() (in module pargopy.stats)}

\begin{fulllineitems}
\phantomsection\label{\detokenize{pargopy:pargopy.stats.date_mode_filter}}\pysiglinewithargsret{\sphinxcode{\sphinxupquote{pargopy.stats.}}\sphinxbfcode{\sphinxupquote{date\_mode\_filter}}}{\emph{mode}, \emph{date}, \emph{itile}}{}
Make the tile filter to choose keep only the chosen mode (‘R’, ‘A’, ‘D’, ‘AD’ or ‘RAD’)
and the profiles under the chosen date (year, month, day)
Return the tile\_extract according to the filters used.
\begin{quote}\begin{description}
\item[{Return type}] \leavevmode
dic

\end{description}\end{quote}

\end{fulllineitems}

\index{generate\_filename() (in module pargopy.stats)}

\begin{fulllineitems}
\phantomsection\label{\detokenize{pargopy:pargopy.stats.generate_filename}}\pysiglinewithargsret{\sphinxcode{\sphinxupquote{pargopy.stats.}}\sphinxbfcode{\sphinxupquote{generate\_filename}}}{\emph{itile}, \emph{typestat}, \emph{reso}, \emph{timeflag}, \emph{date}, \emph{mode}}{}
Generates the filename of the netCDF stat file
\begin{quote}\begin{description}
\item[{Return type}] \leavevmode
str

\end{description}\end{quote}

\end{fulllineitems}

\index{grid\_coordinate() (in module pargopy.stats)}

\begin{fulllineitems}
\phantomsection\label{\detokenize{pargopy:pargopy.stats.grid_coordinate}}\pysiglinewithargsret{\sphinxcode{\sphinxupquote{pargopy.stats.}}\sphinxbfcode{\sphinxupquote{grid\_coordinate}}}{\emph{itile}, \emph{reso}}{}
Returns the coordinates of each point of the grid for a given tile

coordinates are round multiples of reso\_deg
reso sets the grid resolution, typically 0.5deg
\begin{quote}\begin{description}
\item[{Return type}] \leavevmode
numpy.ndarray, numpy.ndarray

\end{description}\end{quote}

\end{fulllineitems}

\index{main() (in module pargopy.stats)}

\begin{fulllineitems}
\phantomsection\label{\detokenize{pargopy:pargopy.stats.main}}\pysiglinewithargsret{\sphinxcode{\sphinxupquote{pargopy.stats.}}\sphinxbfcode{\sphinxupquote{main}}}{\emph{itile}, \emph{typestat}, \emph{reso}, \emph{timeflag}, \emph{date}, \emph{mode}}{}
Main function of stats.py

\end{fulllineitems}

\index{read\_stat\_file() (in module pargopy.stats)}

\begin{fulllineitems}
\phantomsection\label{\detokenize{pargopy:pargopy.stats.read_stat_file}}\pysiglinewithargsret{\sphinxcode{\sphinxupquote{pargopy.stats.}}\sphinxbfcode{\sphinxupquote{read\_stat\_file}}}{\emph{itile}, \emph{typestat}, \emph{reso}, \emph{timeflag}, \emph{date}, \emph{mode}, \emph{var\_choice}}{}
Read statistics into a netcdf file
\begin{quote}\begin{description}
\item[{Return type}] \leavevmode
dict

\end{description}\end{quote}

\end{fulllineitems}

\index{retrieve\_tile\_from\_position() (in module pargopy.stats)}

\begin{fulllineitems}
\phantomsection\label{\detokenize{pargopy:pargopy.stats.retrieve_tile_from_position}}\pysiglinewithargsret{\sphinxcode{\sphinxupquote{pargopy.stats.}}\sphinxbfcode{\sphinxupquote{retrieve\_tile\_from\_position}}}{\emph{lon0}, \emph{lat0}}{}
Return the tile index in which (lon0, lat0) sits
\begin{quote}\begin{description}
\item[{Return type}] \leavevmode
list

\end{description}\end{quote}

\end{fulllineitems}

\index{write\_stat\_file() (in module pargopy.stats)}

\begin{fulllineitems}
\phantomsection\label{\detokenize{pargopy:pargopy.stats.write_stat_file}}\pysiglinewithargsret{\sphinxcode{\sphinxupquote{pargopy.stats.}}\sphinxbfcode{\sphinxupquote{write\_stat\_file}}}{\emph{itile}, \emph{typestat}, \emph{reso}, \emph{timeflag}, \emph{date}, \emph{mode}, \emph{stats\_mode}}{}
Write statistics into a netcdf file
\begin{quote}\begin{description}
\item[{Return type}] \leavevmode
None

\end{description}\end{quote}

\end{fulllineitems}



\section{task\_giver.py}
\label{\detokenize{pargopy:task-giver-py}}\label{\detokenize{pargopy:module-pargopy.task_giver}}\index{pargopy.task\_giver (module)}
Masternslave used for the tiles creation
\index{getavailableslave() (in module pargopy.task\_giver)}

\begin{fulllineitems}
\phantomsection\label{\detokenize{pargopy:pargopy.task_giver.getavailableslave}}\pysiglinewithargsret{\sphinxcode{\sphinxupquote{pargopy.task\_giver.}}\sphinxbfcode{\sphinxupquote{getavailableslave}}}{\emph{slavestate}}{}
Return the index of a slave that is awaiting a task.  A busy slave
has a state == 0. If all slaves are busy then wait until a msg is
received, the msg is sent upon task completion by a slave. Then
determin who sent the msg. The msg is collected in the answer
array. By scanning it, we determine who sent the message.

\end{fulllineitems}

\index{master\_work\_blocking() (in module pargopy.task\_giver)}

\begin{fulllineitems}
\phantomsection\label{\detokenize{pargopy:pargopy.task_giver.master_work_blocking}}\pysiglinewithargsret{\sphinxcode{\sphinxupquote{pargopy.task\_giver.}}\sphinxbfcode{\sphinxupquote{master\_work\_blocking}}}{\emph{nslaves}}{}
Master organizes the work using blocking communications with
slaves

\end{fulllineitems}

\index{master\_work\_nonblocking() (in module pargopy.task\_giver)}

\begin{fulllineitems}
\phantomsection\label{\detokenize{pargopy:pargopy.task_giver.master_work_nonblocking}}\pysiglinewithargsret{\sphinxcode{\sphinxupquote{pargopy.task\_giver.}}\sphinxbfcode{\sphinxupquote{master\_work\_nonblocking}}}{\emph{nslaves}}{}
Main program for master

Master basically supervises things but does no work

\end{fulllineitems}

\index{ordering\_tasks() (in module pargopy.task\_giver)}

\begin{fulllineitems}
\phantomsection\label{\detokenize{pargopy:pargopy.task_giver.ordering_tasks}}\pysiglinewithargsret{\sphinxcode{\sphinxupquote{pargopy.task\_giver.}}\sphinxbfcode{\sphinxupquote{ordering\_tasks}}}{\emph{tasks}}{}
Sort the tasks according to their workload
workload is proportional to size of the tile file
\begin{quote}\begin{description}
\item[{Return type}] \leavevmode
list of int

\end{description}\end{quote}

\end{fulllineitems}

\index{slave\_work\_blocking() (in module pargopy.task\_giver)}

\begin{fulllineitems}
\phantomsection\label{\detokenize{pargopy:pargopy.task_giver.slave_work_blocking}}\pysiglinewithargsret{\sphinxcode{\sphinxupquote{pargopy.task\_giver.}}\sphinxbfcode{\sphinxupquote{slave\_work\_blocking}}}{\emph{islave}}{}
Very simple function for slaves based on blocking communications
with master

\end{fulllineitems}

\index{slave\_work\_nonblocking() (in module pargopy.task\_giver)}

\begin{fulllineitems}
\phantomsection\label{\detokenize{pargopy:pargopy.task_giver.slave_work_nonblocking}}\pysiglinewithargsret{\sphinxcode{\sphinxupquote{pargopy.task\_giver.}}\sphinxbfcode{\sphinxupquote{slave\_work\_nonblocking}}}{\emph{islave}}{}
Main function for slaves.

Slaves enter an infinite loop: keep receiving messages from the
master until reception of ‘done’. Each messages describes the task
to be done. When a new task is received slave treats it. At the
end of it the slave sends a message to the master saying that he
is over, and that he is available for a new task.

\end{fulllineitems}



\section{task\_giver\_stats.py}
\label{\detokenize{pargopy:task-giver-stats-py}}\label{\detokenize{pargopy:module-pargopy.task_giver_stats}}\index{pargopy.task\_giver\_stats (module)}
Masternslave used to create the stats files
\index{getavailableslave() (in module pargopy.task\_giver\_stats)}

\begin{fulllineitems}
\phantomsection\label{\detokenize{pargopy:pargopy.task_giver_stats.getavailableslave}}\pysiglinewithargsret{\sphinxcode{\sphinxupquote{pargopy.task\_giver\_stats.}}\sphinxbfcode{\sphinxupquote{getavailableslave}}}{\emph{slavestate}}{}
Return the index of a slave that is awaiting a task.  A busy slave
has a state == 0. If all slaves are busy then wait until a msg is
received, the msg is sent upon task completion by a slave. Then
determin who sent the msg. The msg is collected in the answer
array. By scanning it, we determine who sent the message.

\end{fulllineitems}

\index{master\_work\_nonblocking() (in module pargopy.task\_giver\_stats)}

\begin{fulllineitems}
\phantomsection\label{\detokenize{pargopy:pargopy.task_giver_stats.master_work_nonblocking}}\pysiglinewithargsret{\sphinxcode{\sphinxupquote{pargopy.task\_giver\_stats.}}\sphinxbfcode{\sphinxupquote{master\_work\_nonblocking}}}{\emph{nslaves}}{}
Main program for master

Master basically supervises things but does no work

\end{fulllineitems}

\index{ordering\_tasks() (in module pargopy.task\_giver\_stats)}

\begin{fulllineitems}
\phantomsection\label{\detokenize{pargopy:pargopy.task_giver_stats.ordering_tasks}}\pysiglinewithargsret{\sphinxcode{\sphinxupquote{pargopy.task\_giver\_stats.}}\sphinxbfcode{\sphinxupquote{ordering\_tasks}}}{\emph{tasks}}{}
Sort the tasks according to their workload
workload is proportional to size of the tile file
\begin{quote}\begin{description}
\item[{Return type}] \leavevmode
list of int

\end{description}\end{quote}

\end{fulllineitems}

\index{slave\_work\_nonblocking() (in module pargopy.task\_giver\_stats)}

\begin{fulllineitems}
\phantomsection\label{\detokenize{pargopy:pargopy.task_giver_stats.slave_work_nonblocking}}\pysiglinewithargsret{\sphinxcode{\sphinxupquote{pargopy.task\_giver\_stats.}}\sphinxbfcode{\sphinxupquote{slave\_work\_nonblocking}}}{\emph{islave}}{}
Main function for slaves.

Slaves enter an infinite loop: keep receiving messages from the
master until reception of ‘done’. Each messages describes the task
to be done. When a new task is received slave treats it. At the
end of it the slave sends a message to the master saying that he
is over, and that he is available for a new task.

\end{fulllineitems}



\section{tile.py}
\label{\detokenize{pargopy:module-pargopy.tile}}\label{\detokenize{pargopy:tile-py}}\index{pargopy.tile (module)}
Created on Wed Apr  4 10:07:26 2018

@author: herry

File used to generate the different tiles of the atlas
\index{generate\_argotiles() (in module pargopy.tile)}

\begin{fulllineitems}
\phantomsection\label{\detokenize{pargopy:pargopy.tile.generate_argotiles}}\pysiglinewithargsret{\sphinxcode{\sphinxupquote{pargopy.tile.}}\sphinxbfcode{\sphinxupquote{generate\_argotiles}}}{}{}
Generate the argodb dictionnary for each tile and save it in
‘argo\%003i’

\end{fulllineitems}

\index{generate\_tile() (in module pargopy.tile)}

\begin{fulllineitems}
\phantomsection\label{\detokenize{pargopy:pargopy.tile.generate_tile}}\pysiglinewithargsret{\sphinxcode{\sphinxupquote{pargopy.tile.}}\sphinxbfcode{\sphinxupquote{generate\_tile}}}{\emph{i}}{}
Interpolate all Argo profiles in tile ‘i’ onto ‘zref’ depths. Save
the result in the ‘tile\%003i.pkl’ file

\end{fulllineitems}

\index{main() (in module pargopy.tile)}

\begin{fulllineitems}
\phantomsection\label{\detokenize{pargopy:pargopy.tile.main}}\pysiglinewithargsret{\sphinxcode{\sphinxupquote{pargopy.tile.}}\sphinxbfcode{\sphinxupquote{main}}}{\emph{itile}}{}
Main function of tile.py

\end{fulllineitems}

\index{plot\_tile() (in module pargopy.tile)}

\begin{fulllineitems}
\phantomsection\label{\detokenize{pargopy:pargopy.tile.plot_tile}}\pysiglinewithargsret{\sphinxcode{\sphinxupquote{pargopy.tile.}}\sphinxbfcode{\sphinxupquote{plot\_tile}}}{\emph{i}}{}
Plots the tiles values (Ti, Si, Ri) with the values non interpolate
\begin{quote}\begin{description}
\item[{Return type}] \leavevmode
None

\end{description}\end{quote}

\end{fulllineitems}



\section{Module contents}
\label{\detokenize{pargopy:module-contents}}\label{\detokenize{pargopy:module-pargopy}}\index{pargopy (module)}

\chapter{Indices and tables}
\label{\detokenize{index:indices-and-tables}}\begin{itemize}
\item {} 
\DUrole{xref,std,std-ref}{genindex}

\item {} 
\DUrole{xref,std,std-ref}{modindex}

\item {} 
\DUrole{xref,std,std-ref}{search}

\end{itemize}


\renewcommand{\indexname}{Python Module Index}
\begin{sphinxtheindex}
\def\bigletter#1{{\Large\sffamily#1}\nopagebreak\vspace{1mm}}
\bigletter{p}
\item {\sphinxstyleindexentry{pargopy}}\sphinxstyleindexpageref{pargopy:\detokenize{module-pargopy}}
\item {\sphinxstyleindexentry{pargopy.argodb}}\sphinxstyleindexpageref{pargopy:\detokenize{module-pargopy.argodb}}
\item {\sphinxstyleindexentry{pargopy.argotools}}\sphinxstyleindexpageref{pargopy:\detokenize{module-pargopy.argotools}}
\item {\sphinxstyleindexentry{pargopy.atlas}}\sphinxstyleindexpageref{pargopy:\detokenize{module-pargopy.atlas}}
\item {\sphinxstyleindexentry{pargopy.check\_decreasing\_pressure}}\sphinxstyleindexpageref{pargopy:\detokenize{module-pargopy.check_decreasing_pressure}}
\item {\sphinxstyleindexentry{pargopy.decorator}}\sphinxstyleindexpageref{pargopy:\detokenize{module-pargopy.decorator}}
\item {\sphinxstyleindexentry{pargopy.general\_tools}}\sphinxstyleindexpageref{pargopy:\detokenize{module-pargopy.general_tools}}
\item {\sphinxstyleindexentry{pargopy.interpolation\_tools}}\sphinxstyleindexpageref{pargopy:\detokenize{module-pargopy.interpolation_tools}}
\item {\sphinxstyleindexentry{pargopy.param}}\sphinxstyleindexpageref{pargopy:\detokenize{module-pargopy.param}}
\item {\sphinxstyleindexentry{pargopy.research\_tools}}\sphinxstyleindexpageref{pargopy:\detokenize{module-pargopy.research_tools}}
\item {\sphinxstyleindexentry{pargopy.stats}}\sphinxstyleindexpageref{pargopy:\detokenize{module-pargopy.stats}}
\item {\sphinxstyleindexentry{pargopy.task\_giver}}\sphinxstyleindexpageref{pargopy:\detokenize{module-pargopy.task_giver}}
\item {\sphinxstyleindexentry{pargopy.task\_giver\_stats}}\sphinxstyleindexpageref{pargopy:\detokenize{module-pargopy.task_giver_stats}}
\item {\sphinxstyleindexentry{pargopy.tile}}\sphinxstyleindexpageref{pargopy:\detokenize{module-pargopy.tile}}
\end{sphinxtheindex}

\renewcommand{\indexname}{Index}
\printindex
\end{document}